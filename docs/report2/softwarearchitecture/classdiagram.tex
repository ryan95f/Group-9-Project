\subsection{Class Diagram}
	\subsubsection{Introduction}
		In many cases, it is desirable to visualise the components of a particular system. This is done to create a better understanding of the role for each of the various components and how they relate to one-another.
		
		The standard convention for these visualisations is through the use of Unified Modeling Language (UML) diagrams. UML provides us with a globally accepted, well-known approach to constructing a diagram of a software system. 
		
		Due to our system's reliance on Django, it is rather infeasible, and unnecessary, of us to construct a diagram for the system as a whole; as it would only really be useful if we recursed into Django's objects.
		
		However, as it happens, the \LaTeX Parser has no direct need to be coupled with Django, so it makes sense to design a diagram to illustrate how we intend the parser to look.
		
	\subsubsection{The \LaTeX Parser}
		The \LaTeX parser's goal is to take a \LaTeX file as input and output the contents of that file as a tree represented using custom objects in Python.
		
		However, as \LaTeX is considered to be Turing complete, the process of correctly parsing such a file is incredibly difficult; far exceeding the scope, and requirements, of this project.
		
		Despite this, it still remains feasible for us to construct a parser capable of parsing files with a \LaTeX-like format; capturing a wide range of semantic definitions in the document.
		
		Taking this into account, we have decided to design the \LaTeX parser in such a way that its capabilities can be extended to a moderate degree.
		
		The concept behind the parser is as following:
		An object, \TeX BookParser, contains the main functionality required for parsing a \LaTeX file. This object is completely oblivious as to how a \LaTeX command's contents/arguments are obtaining; it is only responsible for constructing the tree.
		
		This \TeX BookParser is given an object, NodeBank, which contains several configurations for the parser - along with a few functions for convenience.
		
		Among NodeBank's configurations is a key-value map, which maps an ID to a subclass of LeafNode. This LeafNode is an abstract class which, when subclassed, represents a possible node in the tree.
		
		Each node is used to capture the details of a single particular \LaTeX command. During the parsing of a \LaTeX file, the \TeX BookParser object will check with each node in the NodeBank to determine any matches.
		
		If \TeX BookParser determines that the content of a particular node contains no further nodes, it will create an instance of a special type of node called TextNode. A TextNode is a leaf-node whose contents is only plain-text.
		
		If a \LaTeX command has any arguments, \TeX BookParser will create an instance of a special type of node called ArgumentNode. This new instance of ArgumentNode will be made a child of the initial \LaTeX command's node.
		
		To make creating new nodes easier, we have identified three primary different types of commands in \LaTeX. These will contain the logic required for identifying matches and capturing a \LaTeX command's content and arguments.
		
		These are to be used as a super-class, with the new sub-class being customised as appropriate and added into the NodeBank.
		
		The three different types of commands are as follows:
		\begin{enumerate}
			\item Command Nodes e.g.
				\begin{verbatim}
					\textbf{Example},
					\ref{ExampleReference} 
				\end{verbatim}
			
			\item Environment Nodes e.g.
				\begin{verbatim}
					\begin{itemize} 
						\item First item 
						\item Second item
					\end{itemize},
					
					\begin{center}
						Hello World!
					\end{center},
				\end{verbatim}
			
			\item Level Nodes e.g.
				\begin{verbatim}
					\chapter{Example Chapter}
						\section{Example Section}
							\subsection{Example Subsection}
								\subsubsection{Example Subsubsection}
				\end{verbatim}
		\end{enumerate}
		
		A visual representation of the structure described aboved is listed below, in UML:
		\begin{center}
	\resizebox{.95\textwidth}{!}{
		\begin{tikzpicture}
			\begin{package}{LaTeX Parser}
				\begin{class}[text width=8cm]{TexBookParser}{-5,0}
					\attribute{preamble\_captures : PreambleCapture[0..*]}
					\attribute{nodes : NodeBank}
					
					\operation{\_\_init\_\_(nodes : NodeBank, preamble\_captures : PreambleCapture[0..*])}
					\operation{parse(latex\_text : String) : Node}
					\operation{prepare\_latex\_text(latex\_text : String) : String}
					\operation{read\_preamble(latex\_text : String) : String}
					\operation{read\_body(latex\_text : String) : String}
					\operation{parse\_preamble(preamble : String) : Dictionary}
					\operation{parse\_latex(latex : String) : Node[0..*]}
				\end{class}
				
				\begin{class}[text width=7.5cm]{NodeBank}{3.5,0}
					\attribute{node\_classes : Dictionary}
					\attribute{root\_node\_id : String}
					\attribute{text\_node\_id : String}
					\attribute{argument\_node\_id : String}
					
					\operation{add\_class(node\_class : Node)}
					\operation{get\_class(node\_id : String) : Node}
					\operation{set\_root\_node\_id(node\_id : String) : Node}
					\operation{set\_text\_node\_id(node\_id : String) : Node}
					\operation{set\_argument\_node\_id(node\_id : String) : Node}
					\operation{get\_by\_attr(node\_attr : String) : Node[0.**]}
				\end{class}
			\end{package}
		\end{tikzpicture}
	}
\end{center}
		
\begin{center}
	\resizebox{.95\textwidth}{!}{
		\begin{tikzpicture}
			\begin{package}{Nodes}
				\begin{abstractclass}[text width=8cm]{LeafNode}{0,0}
					\attribute{checkable : Boolean}
					\attribute{allowed\_children : Boolean}
					\attribute{position : Integer}
					
					\operation{\_\_init\_\_()}
					\operation{\_\_str\_\_()}
					\operation{\_\_repr\_\_()}
					\operation{get\_id(latex\_text : String) : Node}
					\operation{check\_latex(latex\_text : String) : Node}
					\operation[0]{to\_html(arguments : *[0..*]) : NodeHTML}
				\end{abstractclass}
				
				\begin{abstractclass}[text width=5cm]{TextNode}{6,-8}
					\inherit{LeafNode}
					
					\attribute{the\_type : String}
					\attribute{content : String}
					
					\operation{\_\_init\_\_(content : String)}
					\operation{\_\_str\_\_()}
					\operation{\_\_repr\_\_()}
					\operation{validate\_content(content : String) : Boolean}
					\operation{set\_content(content : String)}
				\end{abstractclass}
				
				\begin{abstractclass}[text width=8cm]{Node}{-6,-8}
					\inherit{LeafNode}
					
					\attribute{checkable : Boolean}
					\attribute{allowed\_children : Boolean}
					\attribute{position : Integer}
					
					\operation{\_\_init\_\_(children : LeafNode[0..*])}
					\operation{\_\_repr\_\_()}
					\operation{validate\_arguments(arguments : String) : Boolean}
					\operation[0]{to\_html(arguments : *[0..*]) : NodeHTML}
					\operation{add\_children(children : LeafNode[0..*])}
					\operation{add\_child(children : LeafNode)}
				\end{abstractclass}
				
				\begin{abstractclass}[text width=5cm]{ArgumentNode}{3.5,-16}
					\inherit{Node}
					
					\operation{\_\_init\_\_(children : LeafNode[0..*])}
				\end{abstractclass}
				
				\begin{abstractclass}[text width=5cm]{CommandNode}{-7.5,-19}
					\inherit{Node}
					
					\attribute{the\_type : String}
					\attribute{content : String}
					
					\operation{\_\_init\_\_(content : String)}
					\operation{\_\_str\_\_()}
					\operation{\_\_repr\_\_()}
					\operation{get\_start\_regex() : RegexObject}
					\operation{check\_latex(latex : String, nodes : NodeBank) : CheckLatexReturn[0..*]}
				\end{abstractclass}
				
				\begin{abstractclass}[text width=5cm]{EnvironmentNode}{-2,-19}
					\inherit{Node}
					
					\attribute{the\_type : String}
					\attribute{content : String}
					
					\operation{\_\_init\_\_(content : String)}
					\operation{\_\_str\_\_()}
					\operation{\_\_repr\_\_()}
					\operation{get\_start\_regex() : RegexObject}
					\operation{get\_end\_regex() : RegexObject}
					\operation{check\_latex(latex : String, nodes : NodeBank) : CheckLatexReturn[0..*]}
				\end{abstractclass}
			
				\begin{abstractclass}[text width=5cm]{LevelNode}{3.5,-19}
					\inherit{Node}
					
					\attribute{the\_type : String}
					\attribute{content : String}
					
					\operation{\_\_init\_\_(content : String)}
					\operation{\_\_str\_\_()}
					\operation{\_\_repr\_\_()}
					\operation{get\_start\_regex() : RegexObject}
					\operation{check\_latex(latex : String, nodes : NodeBank) : CheckLatexReturn[0..*]}
				\end{abstractclass}
			\end{package}
		\end{tikzpicture}
	}
\end{center}