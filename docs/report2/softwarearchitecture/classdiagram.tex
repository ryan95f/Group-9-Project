\subsection{Class Diagram}
	\subsubsection{Introduction}
		In order to efficiently create a robust and manageable system, it is critical that we consider its individual components and identify how they should interact with one another. Doing so will aid us in avoiding writing code of poor quality; which would be difficult to follow and debug should something go wrong.
		
		The first step to be made is that of identifying the components of our system. This can sometimes be quite a grey area, as it is not always clear where the functionality of a single component should end. To help avoid restricting ourselves too heavily regarding the role of our system's components, we shall first deeply explore the various design patterns available to us and how they could be integrated to solve some of the problems proposed by the system.
	
		However, as discussed previously, the system is to be built using Python's Django Framework.  Therefore a series of abstractions have already been made available to us; greatly simplifying the overall task of developing the system. Given this, we are now able to focus on designing just the core requirements of the system.
		
		Throughout the remainder of this section, we shall discuss our chosen designs for the system; providing the reasoning for our decisions.
	
	\subsubsection{Components}
	
	\subsubsection{Diagram}